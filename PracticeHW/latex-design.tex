
\documentstyle[11pt]{article}
\begin{document}
\begin{center}
{\bf CSE 132A} \\
{\Large Solutions to Practice Problems on Schema Design} \\
\end{center}



\vspace{4mm}
\noindent
{\bf 1.} We apply the lossless join test. The tableau corresponding to the decomposition $\rho$ is:
$$
\begin{array}{llll}
A & B & C & D \\ \hline
a & b & - & - \\
- & b & c & - \\
- & - & c & d\\ \hline
a & b & c & d
\end{array}
$$
After chasing this with respect to  $F = \{B \rightarrow A, C \rightarrow B\}$
the last row becomes $<a,b,c,d>$.

\vspace{4mm}
\noindent
{\bf 2.} The fds $AB \rightarrow C, C \rightarrow E, E \rightarrow C$ are obviously preserved
because each applies to one relation in the decomposition.
Consider $C \rightarrow D$, which does not apply to a single relation.
We compute the closure of $C$ relative to the local fds according to the
algorithm described in class. Initially we have $C$ in the relation $CE$
and in $ABC$. The closure of $C$ is $CED$, so we obtain $E$ within $CE$.
Now $E$ is available in $ADE$. The closure of $E$ is $ECD$ so we obtain $D$
within $ADE$. Since $D$ is now on the list, it is in the closure of $C$
wrt the local fds, so $C \rightarrow D$ is preserved.
It remains to check $AB \rightarrow E$. We now start with $AB$, so we have $AB$ within $ABC$, and $A$ within $ADE$. The closure of $AB$ is $ABCED$ so we obtain $C$ within 
$ABC$. Now we have $C$ within $CE$, and $C^+ = CED$ so we get $E$ within $CE$.
Thus, $AB \rightarrow E$ is also preserved.

\vspace{4mm}
\noindent
{\bf 3.} We first rewrite the fds so that we only have single attributes
on the righthand side:
$$A \rightarrow C, AB \rightarrow C, C \rightarrow I, C \rightarrow D, 
CD \rightarrow I, EC \rightarrow A, EC \rightarrow B, EI \rightarrow C$$
We next look at each of the fds and see if they are redundant.
$A \rightarrow C$ is not, because $A^+$ (wrt the other fds) is $A$.
$AB \rightarrow C$ is clearly redundant, since it is implied by $A \rightarrow C$. 
We eliminate it from the list.
Similarly, $C \rightarrow D$ is not redundant. However, $C \rightarrow I$ is redundant,
and we eliminate it. Next, $CD \rightarrow I$ is not redundant wrt the fds left on the list. Similarly, $EC \rightarrow A$, $EC \rightarrow B$, and $EI \rightarrow C$ are not redundant.
The remaining list of fds is so far:
$$A \rightarrow C, C \rightarrow D, 
CD \rightarrow I, EC \rightarrow A, EC \rightarrow B, EI \rightarrow C.$$

Next, we check for redundant attributes on lefthand sides of fds.
Consider $CD \rightarrow I$. We need to check whether $C$ or $D$ can be eliminated.
$C$ can be eliminated if $D \rightarrow I$ is implied by the fds on the list
(the entire list!). Clearly, $D^+ = D$, so $D \rightarrow I$ is not implied.
Next, $D$ can be eliminated if $C \rightarrow I$ is implied. Now $C^+ = CDI$ so
$C \rightarrow I$ is implied. So $D$ is redundant and
we replace $CD \rightarrow I$ by $C \rightarrow I$ in the list of fds.
It easy to see that there are no redundant attributes in
$$EC \rightarrow A, EC \rightarrow B, EI \rightarrow C$$
so the final minimized set of fds is:
$$A \rightarrow C, C \rightarrow D, 
C \rightarrow I, EC \rightarrow A, EC \rightarrow B, EI \rightarrow C.$$


\vspace{4mm}
\noindent
{\bf 4.}
\begin{itemize}
\item [(a)] $IS$ is a key, i.e. a minimal superkey. Indeed,
$(IS)^+ = ISDBQO$. To see that it is minimal, note that $I$ is not a key
and $S$ is not a key. \\
\item [(b)] $IS$ is the only minimal key. To see this, 
it is enough
to note that any key $K$ must contain $IS$. This is obvious, because
neither $I$ nor $S$ appear on the righthand side of any fd.  \\
\item [(c)] A BCNF decomposition with lossless join obtained 
by the algorithm is:
$$\rho = \{SD,IB,IO,ISQ\}.$$
(See separate bcnf file.)


\item [(d)] It is easy to check that 
$S \rightarrow D, I \rightarrow B, IS \rightarrow Q, B \rightarrow O$ is minimal. Thus,
$\{SD,IB,ISQ,BO\}$ is a 3NF decomposiion which is dependency preserving.
Note that the key $IS$ is a subset of one relation in the decomposition
($ISQ$) so there is no need to add it, and the decomposition also has lossless
join.  
\end{itemize}



\vspace{4mm}
\noindent
{\bf 5.}
\begin{itemize}
\item [(a)] By decomposing $ABCD$ using $D \rightarrow C$ (which violates BCNF
within $ABCD$), we obtain $\{DC, ABD\}$. Clearly, $DC$ is in BCNF
(no violation can occur in a two-attribute relation). In $ABD$, the only
violations could come from fds with a single attribute on the lefthand side.
Thus, it is sufficient to check $A^+,B^+$, and $D^+$ within $ABD$: $A^+
\cap ABD = A, B^+ \cap ABD = B, D^+\cap ABD = D$. So $ABD$ is in BCNF
and the final BCNF decomposition is $\{DC, ABD\}$.
\item [(b)] It is necessary to check preservation of $AB \rightarrow C$ and $B \rightarrow C$.
Our algorithm shows that $AB \rightarrow C$ is preserved, but $B \rightarrow C$ is not.
\item [(c)] First, we rewrite the fds as
$$AB \rightarrow C, AB \rightarrow D, D \rightarrow C, B \rightarrow C.$$
Clearly, $AB \rightarrow C$ is redundant and the remainder set is minimal.
Thus, the 3NF decomposition is $\{ABD, DC, BC\}$. Note that
$ABD$ contains the key $AB$, so there is no need to add a key to the schema.
So the above 3NF decomposition is dependency preserving and has lossless join.
\end{itemize}


\end{document}
